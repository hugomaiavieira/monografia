\chapter{Fundamentação teórica}

TODO: Incrementar muito essa parte. Adicionar uma seção sobre o desenvolvimento tradicional.

\section{Agilismo}
\label{sec:agilismo}

TODO: Rodrigo: [Esta seção pode ser bem expandida (aqui pode ser melhor explicado o lance dos ``equívocos"\ da engenharia de software tradicional). Fale da diferença conceitual entre agilismo e tradicionalismo; explique as curvas de Boehm (Software Engineering Economics, 1981) e de Beck (Extreme Programming Explained, 1999); cite (sem maiores aprofundamentos) a origem do agilismo no Toyota Way; explique iteração e release no contexto de um processo ágil.]

Em 2001 um grupo de dezessete especialistas, reconhecidos pela comunidade como grandes nomes do desenvolvimento software, se reuniram para discutir sobre um crescente conjunto de métodos que vinham surgindo e decidiram usar o termo Agilismo para descrever essa nova geração de métodos ágeis \cite{AgileStory}. Na mesma reunião, eles também escreveram o Manifesto Ágil \cite{AgileManifesto}, delineando um conjunto de valores e princípios que, em resumo, trilham um caminho para a eliminação de documentação e processos desnecessários, buscando a simplicidade, com foco na geração de valor e proximidade com o cliente, além de possibilitar respostas rápidas e eficazes às mudanças.

Pode-se dizer então, que o Desenvolvendo Ágil, ou Agilismo, é um rótulo genérico para os métodos de desenvolvimento de software baseados no Manifesto Ágil \cite{BDDRodrigo}.

% section agilismo (end)

\section{Tipos de teste}
\label{sec:tipos_de_teste}

TODO: Rodrigo: [Explique melhor (e com exemplos) cada tipo de teste, relacionando as definiçoes na literatura ágil com as definições na literatura tradicional.]

\subsection{Testes de unidade}
\label{sub:testes_de_unidade}

Testes de unidade são testes nos quais unidades individuais do sistema são testadas para determinar se estão aptas para uso. Uma unidade é a menor parte testável de uma aplicação. Em programação procedural uma unidade pode ser uma função ou \textit{procedure}. Já em programação orientada a objetos, uma unidade pode ser um método.

% subsection testes_de_unidade (end)

\subsection{Testes de integração}
\label{sub:testes_de_integracao}

Teste de integração testam as integrações do código com o mundo exterior. Podem ser um teste que se comunique através da rede, tenha contato com o sistema de arquivos ou deixe os limites de seu próprio processo \cite{ArtOfAgileDevelopment}.

% subsection testes_de_integracao (end)

\subsection{Testes de aceitação}
\label{sub:testes_de_aceitacao}

Testes de aceitação são especificações para o comportamento e funcionalidade de um sistema. Os testes de aceitação nos mostram se o sistema se comporta corretamente pela perspectiva de um usuário, sem nos dizer nada sobre como o sistema implementa esse comportamento \cite{TestDrivenKoskela}.

% subsection testes_de_aceitacao (end)

% section tipos_de_teste (end)