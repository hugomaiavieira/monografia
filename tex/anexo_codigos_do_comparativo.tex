\chapter{Códigos do comparativo entre texto plano e código puro}
\label{cha:codigo_do_comparativo}

\begin{mycode}{cucumber}%
{Conjunto de especificações em texto plano}{code:bdd_cucumber_comparativo}
Feature: Run a group of scenarios
  As a user
  I want to run a group of scenarios
  In order to compare with pure code specs

  Scenario: Comments should be rendered with Markdown syntax on the tasks page
    Given I am a contributor of "sgtran" project
    And I am authenticated
    And I have a task of "sgtran" project
    When I am on the task page
    And I fill in "comment_content" with "# Some content [link](http://exemplo.com)"
    And I press "Comment"
    Then I should see "Some content" in a "h1" tag
    And I should see "link" in an "a" tag

  Scenario: Sign out
    Given I am an authenticated contributor
    And I am on the dashboard
    When I follow "Sign out"
    Then I should see "Sign in"

  Scenario: Register new project
    Given I am an authenticated contributor
    And I am on the new project page
    When I fill in "Name" with "name-1"
    And I fill in "Description" with "description 1"
    And I press "Save"
    Then I should be the project's owner
    And I should be on the projects board page
    And I should see "name-1 Board"

  Scenario: Try to register projects with erros
    Given I am an authenticated contributor
    And I am on the new project page
    And I press "Save"
    Then I should see "can't be blank"
    And I should not have any project

  Scenario: Edit a project
    Given I am an authenticated contributor
    And I have a project
    When I am on the project edit page
    And I fill in "Name" with "Some_name"
    And I fill in "Description" with "Any description"
    And I press "Save"
    Then I should be on the projects board page
    And I should see "some_name Board"

  Scenario: Delete project
    Given I am an authenticated contributor
    And the following projects:
      | name   | description   |
      | name_1 | description 1 |
      | name_2 | description 2 |
      | name_3 | description 3 |
      | name_4 | description 4 |
    When I delete the 3rd project
    And I am on the dashboard page
    Then I should see "name_1"
    And I should see "name_2"
    And I should see "name_4"
    And I should not see "name_3"

  Scenario: Register a category successfully
    Given I am a contributor of "sgtran" project
    And I am authenticated
    And I am on the "sgtran" categories page
    When I follow "New Category"
    And I fill in "Name" with "Feature"
    And I fill in "Color" with "ffa5a5"
    And I press "Save"
    Then I should see "Category was successfully created."
    And I should see "Feature"
    And I should see "ffa5a5"
    And "sgtran" project should have "1" category

  Scenario Outline: Try to register categories with errors
    Given I am a contributor of "sgtran" project
    And I am authenticated
    And I am on the "sgtran" categories page
    And I have a category with name "Bug" and color "ffa5a5"
    When I follow "New Category"
    And I fill in "Name" with "<name>"
    And I fill in "Color" with "<color>"
    And I press "Save"
    Then I should see "<sentence>"
    And "sgtran" project should have "1" category

    Examples:
    | name    | color  | sentence                   |
    |         | a5d2ff | can't be blank             |
    | Feature |        | can't be blank             |
    | Feature | ffa5a5 | should be uniq for project |
    | Bug     | a5d2ff | should be uniq for project |
    | Bug     | red    | is invalid                 |
    | Bug 1   | a5d2ff | is invalid                 |

  Scenario: Edit a category
    Given I am a contributor of "sgtran" project
    And I am authenticated
    And I have a category with name "Feature" and color "ffa5a5"
    When I am on the category edit page
    And I fill in "Name" with "New feature"
    And I press "Save"
    Then I should see "New feature"
    And I should see "Category was successfully updated."

  Scenario: Register contributors successfully
    Given I am on the new contributor page
    When I fill in "Name" with "Someone of Nothing"
    And I fill in "Username" with "someone"
    And I fill in "E-mail" with "someone@gmail.com"
    And I fill in "Password" with "password"
    And I fill in "Password confirmation" with "password"
    And I press "Sign up"
    Then I should see "You have signed up successfully."

  Scenario Outline: Try to register contributors with errors
    Given I am on the new contributor page
    When I fill in "Name" with "<name>"
    And I fill in "Username" with "<username>"
    And I fill in "E-mail" with "<e-mail>"
    And I fill in "Password" with "password"
    And I fill in "Password confirmation" with "password"
    And I press "Sign up"
    Then I should see "<sentence>"

    Examples:
    | name    | username | e-mail            | sentence       |
    |         | someone  | someone@gmail.com | can't be blank |
    | Someone |          | someone@gmail.com | can't be blank |
    | Someone | someone  |                   | can't be blank |
    | Someone | someone  | someone@nothing   | is invalid     |

  Scenario: Edit a contributor
    Given I am contributor with password "123456" and email "test@test.com"
    And I am authenticated
    And I am on my details edit page
    And I fill in "Username" with "hugomaiavieira"
    And I fill in "Current password" with "123456"
    And I press "Save"
    Then I should see "hugomaiavieira"

  Scenario: Sign in with username successfully
    Given I am contributor with password "123456" and username "someone"
    Given I am on the sign in page
    When I fill in "Login" with "someone"
    And I fill in "Password" with "123456"
    And I press "Sign in"
    Then I should see "Signed in successfully."

  Scenario: Sign in with email successfully
    Given I am contributor with password "123456" and email "someone@example.com"
    Given I am on the sign in page
    When I fill in "Login" with "someone@example.com"
    And I fill in "Password" with "123456"
    And I press "Sign in"
    Then I should see "Signed in successfully."
\end{mycode}

\begin{mycode}{rspec}%
{Conjunto de especificações em código puro}{code:bdd_rspec_comparativo}
require 'spec_helper'

feature 'Authentication' do
  it 'sign out' do
    contributor = Factory.create :contributor
    login(contributor.email, contributor.password)
    click_link 'Sign out'
    page.should have_content 'Sign in'
  end
end

feature 'validates uniqueness of name for contributor' do
  background do
    @hugo = Factory.create :contributor
    @project = Factory.create :project, :name => 'kanban-roots', :owner => @hugo
  end

  scenario 'when edit' do
    login(@hugo.email, @hugo.password)
    visit edit_project_path(@project)
    click_button 'Save'
    page.should have_content 'Project was successfully updated.'
  end

  scenario 'when create' do
    login(@hugo.email, @hugo.password)
    visit new_project_path
    fill_in 'Name', :with => 'kanban-roots'
    click_button 'Save'
    page.should have_content 'you already have a project with this name'
    click_link 'Sign out'

    dudu = Factory.create :contributor
    login(dudu.email, dudu.password)
    visit new_project_path
    fill_in 'Name', :with => 'kanban-roots'
    click_button 'Save'
    page.should have_content 'Project was successfully created.'
  end
end

feature 'manipulate projects' do
  context 'register a new project' do
    before do
      @contributor = Factory.create :contributor
      login @contributor.email, @contributor.password
    end

    it 'successfully' do
      visit new_project_path
      fill_in 'Name', :with => 'name-1'
      fill_in 'Description', :with => 'description 1'
      click_button 'Save'
      page.should have_content 'Project was successfully created.'
      project = Project.all.last
      project.owner.should == @contributor
      current_path.should == project_board_path(project)
      within('h1') do
        page.should have_content 'name-1 Board'
      end
    end

    it 'with errors' do
      visit new_project_path
      click_button 'Save'
      within('form > div[2]') do
        page.should have_content "can't be blank"
      end
      @contributor.projects.should be_empty
    end
  end

  it 'edit a project successfully' do
    contributor = Factory.create :contributor
    project = Factory.create :project, :owner => contributor
    login contributor.email, contributor.password
    visit edit_project_path(project)
    fill_in 'Name', :with => 'Some_name'
    fill_in 'Description', :with => 'Some description'
    click_button 'Save'
    page.should have_content 'Project was successfully updated.'
    current_path.should == project_board_path(project)
    within('h1') do
      page.should have_content 'some_name Board'
    end
  end

  it 'delete a project' do
    contributor = Factory.create :contributor
    projects = []
    5.times do |i|
      projects << (Factory.create :project, :owner => contributor,
                                :name => "name_#{i}", :description => "description #{i}")
    end
    login contributor.email, contributor.password

    visit project_board_path(projects[2])
    click_link 'Admin'
    click_link 'Destroy'
    page.should have_content projects[0].name
    page.should have_content projects[1].name
    page.should_not have_content projects[2].name
    page.should have_content projects[3].name
    page.should have_content projects[4].name
  end
end

feature 'Manage categories' do
  before do
    @project = Factory.create :project
    @contributor = Factory.create :contributor, :contributions => [@project]
    @project.update_attribute(:contributors, [@contributor])
    login @contributor.email, @contributor.password
  end

  context 'register a category' do
    it 'successfully' do
      visit project_categories_path(@project)
      click_link 'New Category'
      fill_in 'Name', :with => 'Feature'
      fill_in 'Color', :with => 'ffa5a5'
      click_button 'Save'
      page.should have_content 'Category was successfully created.'
      page.should have_content 'Feature'
      page.should have_content 'ffa5a5'
      @project.reload.categories.length.should == 1
    end

    context 'with error' do
      before do
        visit project_categories_path(@project)
        click_link 'New Category'
      end

      it 'blank fields' do
        fill_in 'Color', :with => ''
        click_button 'Save'
        within('form > div[2]') { page.should have_content "can't be blank" }
        within('form > div[3]') { page.should have_content "can't be blank" }
      end

      context 'uniq for project' do
        before do
          @category = Factory.create :category, :project => @project,
                                       :color => 'ffa5a5', :name => 'Bug'
        end

        it 'name' do
          fill_in 'Name', :with => 'Bug'
          fill_in 'Color', :with => 'a5d2ff'
          click_button 'Save'
          within('form > div[2]') { page.should have_content 'should be uniq for project' }
        end

        it 'color' do
          fill_in 'Name', :with => 'Feature'
          fill_in 'Color', :with => 'ffa5a5'
          click_button 'Save'
          within('form > div[3]') { page.should have_content 'should be uniq for project' }
        end
      end

      context 'invalid' do
        it 'name' do
          fill_in 'Name', :with => 'Bug 1'
          fill_in 'Color', :with => 'a5d2ff'
          click_button 'Save'
          within('form > div[2]') { page.should have_content 'is invalid' }
        end

        it 'color' do
          fill_in 'Name', :with => 'Bug'
          fill_in 'Color', :with => 'red'
          click_button 'Save'
          within('form > div[3]') { page.should have_content 'is invalid' }
        end
      end
    end
  end

  it 'edit a category' do
    category = Factory.create :category, :project => @project,
                                :name => 'Feature', :color => 'ffa5a5'
    visit edit_project_category_path(@project, category)
    fill_in 'Name', :with => 'New feature'
    click_button 'Save'
    page.should have_content 'New feature'
    page.should have_content 'Category was successfully updated.'
  end
end

feature 'Manipulate contributors' do
  context 'register a contributor' do
    it 'successfully' do
      visit new_contributor_registration_path
      fill_in 'Name', :with => 'Someone of Nothing'
      fill_in 'Username', :with => 'someone'
      fill_in 'E-mail', :with => 'someone@gmail.com'
      fill_in 'Password', :with => 'password'
      fill_in 'Password confirmation', :with => 'password'
      click_button 'Sign up'
      page.should have_content 'You have signed up successfully.'
    end

    context 'with errors' do
      it 'blank fields' do
        visit new_contributor_registration_path
        click_button 'Sign up'
        within('form > div[2]') { page.should have_content "can't be blank"}
        within('form > div[3]') { page.should have_content "can't be blank"}
        within('form > div[4]') { page.should have_content "can't be blank"}
        within('form > div[5]') { page.should have_content "can't be blank"}
      end

      it 'invalid' do
        visit new_contributor_registration_path
        fill_in 'E-mail', :with => 'someone@nothing'
        click_button 'Sign up'
        within('form > div[4]') { page.should have_content "is invalid"}
      end
    end
  end

  context 'edit a contributor' do
    before do
      @contributor = Factory.create :contributor
      login @contributor.email, @contributor.password
    end

    it 'successfully' do
      visit edit_contributor_registration_path
      fill_in 'Username', :with => 'hugomaiavieira'
      fill_in 'Current password', :with => '123456'
      click_button 'Save'
      page.should have_content 'hugomaiavieira'
    end

    it 'current password invalid' do
      visit edit_contributor_registration_path
      fill_in 'Username', :with => 'hugomaiavieira'
      fill_in 'Current password', :with => '123456789'
      click_button 'Save'
      within('form > div[7]') { page.should have_content 'is invalid' }
    end
  end
end

feature 'Sign in successfully' do
  it 'with username' do
    contributor = Factory.create :contributor, :username => 'someone'
    login contributor.username, contributor.password
    page.should have_content 'Signed in successfully.'
  end

  it 'with e-mail' do
    contributor = Factory.create :contributor, :email => 'someone@example.com'
    login contributor.email, contributor.password
    page.should have_content 'Signed in successfully.'
  end
end
\end{mycode}