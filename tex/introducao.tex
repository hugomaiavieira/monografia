\chapter{Introdução}

Com o passar dos anos, principalmente em uma época onde a tecnologia e os negócios mudam e evoluem de modo extremamente rápido, o mercado demanda e espera software inovadores e de alta qualidade, que sejam adequados a suas necessidades $-$ e o mais rápido possível \cite{TheBusinessOfInnovation}.

O desenvolvimento ágil de software, que neste ano de 2012 completa 11 anos, foi elaborado \cite{AgileManifesto} visando atender à estas expectativas do mercado, focando o processo de desenvolvimento nas pessoas e abraçando as mudanças que  naturalmente surgem durante a construção do software. De acordo com \citeonline{PMNetworkFailureDrop}, o \textit{Chaos Manifesto 2011}\footnote{O \textit{Chaos Manifesto} é uma pesquisa bienal realizada pelo \textit{The Standish Group} e teve início em 1994. As pesquisas publicadas em um ano representam os dados do ano anterior.} mostra que os resultados de 2010 representam, desde sua primeira edição, a maior taxa de sucesso nos projetos de desenvolvimento de software, que aumentou de 32\% em 2008 para 37\% em 2010. Segundo \citeonline{ResumoChaosReport}, o \textit{The Standish Group} conclui que uma das razões para o aumento da taxa de sucesso foi a utilização das metodologias ágeis, que cresce a uma taxa de 22\% CAGR\footnote{\href{http://en.wikipedia.org/wiki/Compound_annual_growth_rate} {Compound annual growth rate}} e hoje são adotados em 9\% de todos os projetos de Tecnologia da Informação em andamento e em 29\% dos novos projetos.

Como a adoção do desenvolvimento ágil é crescente nos últimos anos, diversos métodos e técnicas vem sendo desenvolvidos tendo como base os princípios e valores ágeis \cite{BDDRodrigo}, principalmente relacionadas ao teste de software e que serão abordadas no presente trabalho, sendo elas o Desenvolvimento Guiado por Testes (do inglês, \textit{Test-Driven Development} - TDD)\nomenclature{TDD}{Test-Driven Development}, Desenvolvimento Guiado por Comportamento (do inglês, \textit{Behaviour-Driven Development - BDD})\nomenclature{BDD}{Behaviour-Driven Development}, Integração Contínua e Dublês de Teste. Sendo técnicas emergentes, ainda são pouco discutidas no meio acadêmico, este trabalho pretende contribuir com esta discussão e com a introdução destas técnicas emergentes na academia.


\section{Justificativas e objetivos}

Existem poucos trabalhos acadêmicos no que se refere a técnicas emergentes de testes de software, como notado também por \citeonline{BDDSolis}, que em 2011 foram os primeiros a ter um artigo publicado sobre Desenvolvimento Guiado por Comportamento (BDD), pela \textit{IEEE Computer Society}.

Sendo assim, o objetivo do presente trabalho é contribuir com a introdução de técnicas e discussões surgidas no meio empresarial para a academia, além de agregar conhecimentos, ora dispersos e difusos, sobre as diferentes abordagens, possibilidades e pontos em aberto no emprego de tais técnicas.

As técnicas que serão abordadas neste trabalho tem seu conceito definido no mercado e evoluem através da evolução das ferramentas que os implementam. O fluxo de evolução é o seguinte:

\begin{figure}[h]
  \center
  \caption{Fluxo de evolução dos conceitos e ferramentas}
  \includegraphics[scale=0.60]{images/fluxo-conceito-ferramenta}
  \label{img:fluxo_conceito_ferramenta}
\end{figure}

Primeiramente cria-se um conceito e, em seguida, uma ferramenta que o implemente. Com base na utilização e observação desta ferramenta, há uma percepção de novas necessidades, fazendo que os conceitos evoluam e novas ferramentas sejam criadas.

Desta forma, a evolução das ferramentas e a evolução conceitual estão intimamente ligadas. No texto original sobre BDD \cite{IntroducingBDD} já se encontra isto. A inspiração para a criação de BDD foi uma ferramenta chamada \textit{AgileDox}\footnote{Mais informações em \url{http://agiledox.sourceforge.net}}, que fez o autor antever as novas possibilidades que cristalizou no conceito de BDD. Justamente por esta característica dinâmica, a informação e os diferentes conceitos estão dispersos, pois nunca foram sistematizados, ficando em uma espécie de ``inteligência coletiva"\ da comunidade de desenvolvimento ágil.

Além disso, como o presente trabalho está no contexto dos métodos ágeis, e nestes as partes conceitual e prática formam um todo inseparável, para toda técnica abordada serão utilizados exemplos mostrando código de um projeto real.

\section{Metodologia}

Será feita uma explanação sobre cada técnica e uma discussão comparando as diferentes abordagens, possibilidades e pontos em aberto no emprego de cada técnica.

Como base para a discussão, será utilizado o kanban-roots\footnote{\url{http://github.com/hugomaiavieira/kanban-roots}}, que foi desenvolvido pelo autor do presente trabalho utilizando todas as técnicas abordadas neste, possibilitando desta forma a obtenção de dados e experiências da utilização das técnicas em um projeto real.

O kanban-roots é um kanban\footnote{O termo tem origem no sistema Toyota de produção, onde kanban é a maneira como é coordenado o fluxo de peças na cadeia de suprimentos  \cite{AMaquinaQueMudouOMundo}. No contexto do presente trabalho, kanban é um quadro para visualização do fluxo de trabalho (tarefas) em um projeto.} online para auxiliar a organização e acompanhamento das tarefas em um projeto, sendo especialmente interessante para projetos \textit{opensouce} ou, de modo geral, para projetos com equipes geograficamente distribuídas.

Na figura \ref{img:tela_kaban_roots} pode ser visto um \textit{screen shot} do kanban de um projeto no kanban-roots.

O kanban é muito utilizado em metodologias ágeis como XP e Scrum como um quadro para visualização do fluxo de tarefas nas iterações de um projeto (mais na Seção \ref{sub:agilismo}). As tarefas inicialmente são posicionadas na divisão \textbf{\textit{Backlog}} até serem escolhidas para fazer parte de uma iteração. Nesse momento, as tarefas escolhidas vão para a divisão \textbf{\textit{To Do}} até que sejam escolhidas por um desenvolvedor para serem implementadas, passando para a divisão \textbf{\textit{Doing}}. Após o desenvolvedor concluir a tarefa, esta vai para a divisão \textbf{\textit{Done}}. Dessa maneira, todos os participantes tem a possibilidade de ver como está o andamento da iteração, além de ser possível acompanhar o que cada integrante do projeto está está fazendo naquele instante.

O kanban-roots já está em produção e vem sendo testado e utilizado com sucesso por algumas pessoas em empresas do Brasil como a Algorich, Voxline, Mandic, Quatix, mas também estrangeiras como a infoPiiaf e Free.fr (França), Ginzametrics (Estados Unidos), Osube (China), Centah (Canadá), Podmoskovie.info (Rússia), EvoEnergy (Inglaterra), Forgotten Labs (Lituânia), entre outros.

\begin{figure}[h]
  \center
  \caption{Tela do kanban de um projeto no kanban-roots}
  \includegraphics[scale=0.45]{images/kanban-roots}
  \label{img:tela_kaban_roots}
\end{figure}

Todos os trechos de código apresentados neste trabalho são trechos retirados do kanban-roots e a primeira linha de cada trecho sempre será um comentário informando o nome do arquivo original em que o dado trecho se encontra.

\section{Ferramentas utilizadas}

Para o desenvolvimento do kanban-roots foram utilizadas diversas ferramentas, sendo importante citar em que contexto e momento cada uma delas é utilizada.

Como base para o desenvolvimento, foi utilizado o \textit{framework web} Ruby On Rails\footnote{\url{http://rubyonrails.org}}. Para os testes de unidade apresentados na Seção \ref{sub:tdd} foi utilizado o Test::Unit\footnote{\url{http://test-unit.rubyforge.org/}}. Já na Seção \ref{sub:bdd} é utilizado o Rspec\footnote{\url{http://rspec.info/}} para testes unitários, testes de aceitação e dublês de teste. Ainda na Seção \ref{sub:bdd} também foi utilizado o Cucumber\footnote{\url{http://cukes.info/}} para testes de aceitação. Além dessas ferramentas, também foi utilizado o FactoryGirl\footnote{\url{https://github.com/thoughtbot/factory_girl}} para \textit{fixtures replacement} em todos os momentos em que se fez necessário.

\section{Trabalhos relacionados} % (fold)
\label{sec:trabalhos_relacionados}

\textbf{TODO: puxar os trabalhos relacionados de todas as técnicas abordadas}

Não existem muitos trabalhos acadêmicos relacionados às técnicas abordadas no presente trabalho, e todos eles abordam o tema apenas conceitualmente, deixando uma lacuna em termos de contextualização.

\citeonline{BDDSolis} apresentam algumas das principais características do \textit{Behaviour-Driven Development} (BDD), tendo como base uma pequena quantidade de estudo publicados sobre o tema e as ferramentas existentes para a utilização da técnica.

Sobre \textit{Test-Driven Development} (TDD), praticamente são apenas encontrados estudos empíricos, com estudos de caso na academia e na indústria, que discutem a efetividade do uso do TDD. Na Seção \ref{sub:a_efetividade_do_tdd} será feita uma análise destes estudos.

% section trabalhos_relacionados (end)