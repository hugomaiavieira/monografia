\chapter{Introdução}

A tecnologia e os negócios mudam e evoluem de modo extremamente rápido hoje e o mercado demanda e espera software inovadores e de alta qualidade, que sejam adequados a suas necessidades e desenvolvidos o mais rápido possível \cite{TheBusinessOfInnovation}.

O desenvolvimento ágil de software, que neste ano de 2012 completa 11 anos, foi elaborado visando atender estas expectativas do mercado \cite{AgileManifesto}, focando o processo de desenvolvimento nas pessoas e abraçando as mudanças que naturalmente surgem durante o desenvolvimento do software.

Como a adoção do desenvolvimento ágil é crescente nos últimos anos \cite{ResumoChaosReport}, diversos métodos e técnicas vem sendo desenvolvidos, tendo como base os princípios e valores ágeis \cite{BDDRodrigo}, principalmente relacionadas ao teste de software. Contudo, estes métodos e técnicas continuam em evolução, tendo alguns de seus pontos ainda em aberto.

Este trabalho pretende contextualizar e discutir a utilização de técnicas emergentes de teste de software, aqui definidas como Desenvolvimento Guiado por Testes (do inglês, \textit{Test-Driven Development} - TDD)\nomenclature{TDD}{Test-Driven Development}, Desenvolvimento Guiado por Comportamento (do inglês, \textit{Behaviour-Driven Development - BDD})\nomenclature{BDD}{Behaviour-Driven Development}, Integração Contínua e Dublês de Teste. Para esta contextualização, será utilizado um sistema web desenvolvido pelo autor aplicando tais técnicas: o kanban-roots.

\section{Justificativas e objetivos}

São encontrados alguns poucos trabalhos relacionados ao tema abordado nesta monografia, como constata \citeonline{BDDSolis}. Sendo assim, o objetivo do presente trabalho é oferecer um estudo teórico-prático das técnicas emergentes na área de teste de software com metodologias ágeis de desenvolvimento, visando de agregar conhecimentos ora dispersos e difusos, sobre as diferentes abordagens, possibilidades e pontos em aberto no emprego de tais técnicas.

As técnicas que serão abordadas neste trabalho tiveram seu conceito definido no mercado e evoluem através da evolução das ferramentas que as implementam. Na figura \ref{img:fluxo_conceito_ferramenta} pode-se visualizar o fluxo de evolução onde primeiramente cria-se um conceito e, em seguida, uma ferramenta que o implemente. Com base na utilização e observação desta ferramenta, há uma percepção de novas necessidades, fazendo que os conceitos evoluam e novas ferramentas sejam criadas.

\begin{figure}[h]
  \center
  \caption{Fluxo de evolução dos conceitos e ferramentas}
  \includegraphics[scale=0.60]{images/fluxo-conceito-ferramenta}
  \label{img:fluxo_conceito_ferramenta}
\end{figure}

Desta forma, a evolução das ferramentas e a evolução conceitual estão intimamente ligadas. No texto original sobre BDD \cite{IntroducingBDD} já se encontra isto. A inspiração para a criação de BDD foi uma ferramenta chamada \textit{AgileDox}\footnote{Mais informações em \url{http://agiledox.sourceforge.net}}, que fez o autor da mesma antever as novas possibilidades que então cristalizou no conceito de BDD. Justamente por esta característica dinâmica de evolução, as informações e os diferentes conceitos sobre as técnicas estão dispersos, pois nunca foram sistematizados, ficando em uma espécie de ``inteligência coletiva"\ da comunidade de desenvolvimento ágil.

Além disso, como o presente trabalho está no contexto dos métodos ágeis, e nestes as partes conceitual e prática formam um todo inseparável, para toda técnica abordada serão exemplificados os conceitos usando o código de uma ferramenta web, denominada kanban-roots, desenvolvida pelo próprio autor do presente trabalho, utilizando práticas ágeis de desenvolvimento, especificamente Desenvolvimento guiado por Testes (TDD) e Desenvolvimento guiado por Comportamento (BDD).

\section{Metodologia}

Para atingirmos os objetivos propostos neste trabalho, será feita uma explanação sobre cada uma das técnicas emergentes estudadas, aqui definidas como:

\begin{itemize}
  \item Desenvolvimento guiado por testes (TDD, do inglês \textit{Test-Driven Development})
  \item Desenvolvimento guiado por comportamento (BDD, do inglês \textit{Behaviour-Driven Development}
  \item Integração contínua
  \item Dublês de teste
\end{itemize}

Serão comparadas as diferentes abordagens, possibilidades, pontos em aberto no emprego de cada técnica, observando a aplicabilidade de cada uma delas e a eficiência respectiva.

Como base para a discussão, será utilizado o kanban-roots\footnote{\url{http://github.com/hugomaiavieira/kanban-roots}}, que será apresentado na Seção \ref{sec:kanban_roots}. O kanban-roots é um projeto que foi desenvolvido pelo autor do presente trabalho utilizando todas as técnicas abordadas neste, possibilitando desta forma a obtenção de dados e experiências da utilização das técnicas em um projeto real.

Todos os trechos de código apresentados neste trabalho são trechos retirados do kanban-roots e a primeira linha de cada trecho sempre será um comentário informando o nome do arquivo original em que o referido trecho se encontra.

\section{Ferramentas utilizadas}

Para o desenvolvimento do kanban-roots foram utilizadas diversas ferramentas, sendo importante citar em que contexto e momento cada uma delas é utilizada.

Como base para o desenvolvimento, foi utilizado o \textit{framework web} Ruby On Rails\footnote{\url{http://rubyonrails.org}} que é escrito utilizando a linguagem de programação Ruby. Para os testes de unidade apresentados na Seção \ref{sub:tdd} foi utilizado o Test::Unit\footnote{\url{http://test-unit.rubyforge.org/}}. Já na Seção \ref{sub:bdd} é utilizado o Rspec\footnote{\url{http://rspec.info/}} para testes unitários, testes de aceitação e dublês de teste. Ainda na Seção \ref{sub:bdd} também foi utilizado o Cucumber\footnote{\url{http://cukes.info/}} para testes de aceitação. Além dessas ferramentas, também foi utilizado o FactoryGirl\footnote{\url{https://github.com/thoughtbot/factory_girl}} para \textit{fixtures replacement} em todos os momentos em que se fez necessário.

\section{Organização}
\label{sec:organizacao}

No Capítulo 2 será fundamentado os diferentes tipos de teste de software bem como as técnicas emergentes abordadas no presente trabalho. No Capítulo 3 serão contextualizadas as técnicas emergentes de teste de software, discutindo suas abordagens e pontos em aberto. No Capítulo 4 são apresentadas as conclusões obtidas e os trabalhos futuros.