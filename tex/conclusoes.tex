\chapter{Conclusões}
\label{cha:conclusoes}

O presente trabalho discutiu e contextualizou as mais importantes técnicas de emergentes de teste de software, sendo elas TDD, BDD, Integração Contínua e Dublês de Teste, fazendo uma ``revisão crítica"\ das mesmas e construindo conhecimento em cima de informações ainda dispersas e não sistematizadas.

Discutiram-se aqui temas de pouca ou nenhuma sistematização acadêmica. Na Seção \ref{sub:a_ubiquidade_do_tdd} foi discutida a pertinência da utilização de técnicas de \textit{test-first programming} $-$ como TDD $-$ em diferentes contextos. Na Seção \ref{sub:a_relacao_entre_bdd_e_tdd} foi mostrada a relação entre TDD e BDD e como essa diferença também está relacionada com a legibilidade dos testes automatizados. Já na Seção \ref{ssub:pontos_em_aberto} foram levantados os pontos em aberto acerca dos modelos de escrita de testes de aceitação, sendo feitas algumas reflexões sobre estes pontos. Na na Seção \ref{sub:sincrona_x_assincrona} foram expostas as diferenças entre os modelos de integração contínua, mostrando que a integração contínua síncrona é preferencial, porém nem sempre possível. Ainda, na Seção \ref{sub:contextualizando_o_stub}, foram discutidos os benefícios e malefícios do uso dos dublês de teste.

Além disto, todas as discussões foram contextualizada com exemplos originados de uma aplicação real, o projeto kanban-roots.

Assim, este trabalho contribuiu com a introdução de tais ideias na academia, abordando as mesmas de forma contextualizada.

\section{Trabalhos futuros}
\label{sec:trabalhos_futuros}

O ciclo apresentado na figura \ref{img:fluxo_conceito_ferramenta} continua sendo alimentado pelas técnicas e ferramentas apresentadas no presente trabalho, fazendo com que novas técnicas comecem a emergir e estas novas técnicas ainda embrionárias podem ser abordadas em trabalhos futuros. Algumas delas são o Teste Contínuo (do inglês, \textit{Continuous Testing}), a Entrega Contínua (do inglês, \textit{Continuous Deploy}) e a Integração Contínua com servidor de \textit{Staging}.