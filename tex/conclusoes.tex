\chapter{Conclusões} % (fold)
\label{cha:conclusoes}

\textbf{TODO: refazer todo esse capítulo}

O presente trabalho discutiu e contextualizou as mais importantes técnicas de emergentes de teste de software, sendo elas TDD, BDD, Integração Contínua e Dublês de Teste, fazendo uma ``revisão crítica"\ das mesmas e construindo conhecimento em cima de informações ainda dispersas e não sistematizadas.

Discutiram-se aqui, além de uma revisão bibliográfica de agilismo, temas de pouca ou nenhuma sistematização acadêmica, como legibilidade de testes automatizados, comparação entre modelos de escrita de testes de aceitação, a pertinência da utilização de técnicas de \textit{test-first programming} como TDD em diferentes contextos, problemas no uso de dublês de teste, entre outros. Além disto, a discussão foi contextualizada com exemplos originados de uma aplicação real, o projeto kanban-roots.

Por serem técnicas e conceitos que emergiram no meio empresarial, tendo recentemente uma maior aceitação e difusão, ainda não existe muita reflexão acadêmica sobre as mesmas. Desta forma, este trabalho contribuiu com a introdução de tais ideias na academia, abordando as mesmas de forma contextualizada.

% chapter conclusoes (end)