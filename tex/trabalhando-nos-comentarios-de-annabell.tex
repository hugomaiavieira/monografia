2) Ver com rodrigo o que ele acha.

  Na seção Justificativas no 2 parágrafo
  ======================================

  As técnicas, que no capitulo \ref{cha:fundamentacao_teorica} serão explicitadas, tem seu conceito definido no mercado [TODO: remover o "mercado" e colocar "em várias referências tais como 'livros de agile e etc'"] e evoluem através da evolução das ferramentas que os implementam. Na Figura 1 pode-se visualizar o fluxo de evolução onde primeiramente cria-se um conceito ...


3)

  No 5 parágrafo no seria: “as ferramentas e os diferentes conceitos estão...” em vez de informação?
  ============================================================================

  [TODO: melhorar esse parágrafo, removendo o 'informações' e deixando mais claro o objetivo]


5)

  Para atingirmos os objetivos propostos neste trabalho, será feita uma explanação sobre cada uma das técnicas emergentes estudadas, aqui definidas como:

  \begin{itemize}
    \item Desenvolvimento guiado por testes (TDD, do inglês \textit{Test-Driven Development})
    \item Desenvolvimento guiado por comportamento (BDD, do inglês \textit{Behaviour-Driven Development}
    \item Integração contínua
    \item Dublês de teste
  \end{itemize}

  Comparando as diferentes abordagens, possibilidades, pontos em aberto no emprego de cada técnica assim como observando a aplicabilidade de cada uma delas e a eficiência respectiva.

  A ideia inicial para desenvolver este trabalho foi causada pela vivencia do autor no estagio desenvolvido durante o período de estudante.....AQUI É COM VOCÊ HUGO..... Tivemos dificuldades na compreensão dos conceitos.... [HUGO: ainda não consegui pensar o que eu escreveria aqui.]

  [HUGO: A parte que explico de que se trata o kanban-roots seria eliminada, permaneceria aqui, ou entraria em um outro tópico?]
  [TODO: mover a explicação para um novo tópico em fundamentação teórica]


6)

  DESENVOLVIMENTO/REFERENCIAL TEORICO

  Teste de Software é considerado por [Sommerville, 2007] uma técnica dinâmica de verificação e validação. Onde executamos uma implementação do software desenvolvido, apoiados em dados de teste, examinamos a saída e seu comportamento operacional para verificarmos se o seu desempenho está como solicitado/conforme necessário.

  Podemos inserir uma figura do processo de desenvolvimento iterativo, dado por Sommerville na figura 17.1 da pagina 261.

  [HUGO: em que seção essa parte entraria? Na fundamentação quando falamos de teste de software?]

  [TODO: Remover o tópico trabalhos relacionados.]


7)

  Verificar qual a parte conceitual dos métodos ágeis (texto definido na página 12). [HUGO: não entendi]


[TODO: Adicionar na introdução o tópico Organização uq vai falar como a monografia está organizada, no capito tal será apresentado no sei o que n sei o que, etc.]


[TODO: na fundamentação teórica, melhorar todo o fluxo das informações retirando os títulos 'desenvolvimento de softweare' e 'agilismo' de tópicos e dando uma introdução melhor ao tópico 'teste de software']

[TODO: melhorar a fundamentação do teste de integração, falar realmente o que é um teste de integração, como foi feito com teste de unidade]

[TODO: puxar o exemplo de teste de integração da fundamentação para a contextualização.]

[TODO: puxar a parte sobre o kanban que está na metodologia como primeiro tópico do capitulo Contextualização]

[TODO: no último paragrafo da contextualização de tdd, explicar melhor porque atende os requisitos e está limpo e coberto de testes,"como pode ser visto no Código 11"]