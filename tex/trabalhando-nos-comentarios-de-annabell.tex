1)

  ANTES
  =====

  Como a adoção do desenvolvimento ágil é crescente nos últimos anos, diversos métodos e técnicas vem sendo desenvolvidos tendo como base os princípios e valores ágeis \cite{BDDRodrigo}, principalmente relacionadas ao teste de software e que serão abordadas no presente trabalho, sendo elas o Desenvolvimento Guiado por Testes (do inglês, \textit{Test-Driven Development} - TDD)\nomenclature{TDD}{Test-Driven Development}, Desenvolvimento Guiado por Comportamento (do inglês, \textit{Behaviour-Driven Development - BDD})\nomenclature{BDD}{Behaviour-Driven Development}, Integração Contínua e Dublês de Teste. Sendo técnicas emergentes, ainda são pouco discutidas no meio acadêmico, este trabalho pretende contribuir com esta discussão e com a introdução destas técnicas emergentes na academia.


  Instrodução, 3 parágrafo: Modificar o texto.
  ============================================

  Após a frase teste de software, podemos inserir: Este trabalho pretende contribuir e contextualizar a utilização de práticas emergentes, aqui definidas como (colocar por extenso) TDD, Desenvolvimento Guiado pó Comportamento (DGC ou do inglês BDD),??;  que focam o modelo de desenvolvimento orientado ao teste, mostrando uma ferramenta on-line desenvolvida para este fim: O Kanban-Roots.



  DEPOIS
  ======

  Como a adoção do desenvolvimento ágil é crescente nos últimos anos, diversos métodos e técnicas vem sendo desenvolvidos tendo como base os princípios e valores ágeis \cite{BDDRodrigo}, principalmente relacionadas ao teste de software. Este trabalho pretende contribuir e contextualizar a utilização de técnicas emergentes de teste de software, aqui definidas como Desenvolvimento Guiado por Testes (do inglês, \textit{Test-Driven Development} - TDD)\nomenclature{TDD}{Test-Driven Development}, Desenvolvimento Guiado por Comportamento (do inglês, \textit{Behaviour-Driven Development - BDD})\nomenclature{BDD}{Behaviour-Driven Development}, Integração Contínua e Dublês de Teste. Para esta contextualização, sera utilizada um sistema web desenvolvido utilizando tais técnicas: O Kanban-roots.


2)

  ANTES
  =====

  Existem poucos trabalhos acadêmicos no que se refere a técnicas emergentes de testes de software, como notado também por \citeonline{BDDSolis}, que em 2011 foram os primeiros a ter um artigo publicado sobre Desenvolvimento Guiado por Comportamento (BDD), pela \textit{IEEE Computer Society}.

  Sendo assim, o objetivo do presente trabalho é contribuir com a introdução de técnicas e discussões surgidas no meio empresarial para a academia, além de agregar conhecimentos, ora dispersos e difusos, sobre as diferentes abordagens, possibilidades e pontos em aberto no emprego de tais técnicas.

  As técnicas que serão abordadas neste trabalho tem seu conceito definido no mercado e evoluem através da evolução das ferramentas que os implementam. O fluxo de evolução é o seguinte:


  Na seção Justificativas no 1 parágrafo, talvez possamos escrever algo assim como:
  =============================================================================

  Na literatura são encontrados alguns poucos trabalhos relacionados ao tema que estamos querendo abordar nesta monografia, como constata [Solis, 2011]. Sendo assim, o objetivo do presente trabalho é oferecer um estudo teórico prático das novas tendências/técnicas emergentes na área de teste de software com metodologia/desenvolvimento ágil para o desenvolvimento/a criação de sistemas/softwares. Em vistas de agregar conhecimentos, ora dispersos e difusos, sobre as diferentes abordagens, possibilidades e pontos em aberto no emprego de tais técnicas.

  Como objetivo secundário apresentaremos a ferramenta criada neste projeto de trabalho, denominada Kanban-Roots para exemplificar o uso das técnicas emergentes de desenvolvimento ágil que serão apresentadas neste trabalho de monografia.

  No 3 parágrafo devia incluir:
  =============================

  As técnicas, que no capitulo ? serão abordadas/explicitadas, tem seu .... que os implementa. Na Figura 1 pode-se visualizar o fluxo de evolução onde primeiramente cria-se um conceito......


  DEPOIS
  ======

  Na literatura são encontrados alguns poucos trabalhos relacionados ao tema que estamos querendo abordar nesta monografia, como constata \citeonline{BDDSolis}. Sendo assim, o objetivo do presente trabalho é oferecer um estudo teórico prático das novas tendências/técnicas emergentes na área de teste de software com metodologias ágeis de desenvolvimento. Em vistas de agregar conhecimentos, ora dispersos e difusos, sobre as diferentes abordagens, possibilidades e pontos em aberto no emprego de tais técnicas.

  Como objetivo secundário apresentaremos a ferramenta criada neste projeto de trabalho, denominada Kanban-Roots para exemplificar o uso das técnicas emergentes de desenvolvimento ágil que serão apresentadas neste trabalho de monografia.

  As técnicas, que no capitulo \ref{cha:fundamentacao_teorica} serão explicitadas, tem seu conceito definido no mercado e evoluem através da evolução das ferramentas que os implementam. Na Figura 1 pode-se visualizar o fluxo de evolução onde primeiramente cria-se um conceito ...


3)

  No 5 parágrafo no seria: “as ferramentas e os diferentes conceitos estão...” em vez de informação?
  ============================================================================

  Não, é informação mesmo.



4)

  ANTES
  =====

  Além disso, como o presente trabalho está no contexto dos métodos ágeis, e nestes as partes conceitual e prática formam um todo inseparável, para toda técnica abordada serão utilizados exemplos mostrando código de um projeto real.


  No último parágrafo da seção após:
  ==================================

  “...um todo inseparável, para toda técnica abordada serão exemplificados os conceitos usando uma ferramenta on-line desenvolvida pelo próprio autor, desenvolvida usando práticas ágeis de desenvolvimento, especificamente Desenvolvimento guiado por Testes (TDD) e Desenvolvimento guiado por Comportamento (BDD); denominada Kanban-roots.


  DEPOIS
  ======

  Além disso, como o presente trabalho está no contexto dos métodos ágeis, e nestes as partes conceitual e prática formam um todo inseparável, para toda técnica abordada serão exemplificados os conceitos usando o código de uma ferramenta web desenvolvida pelo próprio autor, utilizando práticas ágeis de desenvolvimento, especificamente Desenvolvimento guiado por Testes (TDD) e Desenvolvimento guiado por Comportamento (BDD); denominada Kanban-roots.


5)

  Para atingirmos os objetivos propostos neste trabalho, será feita uma explanação sobre cada uma das técnicas emergentes estudadas, aqui definidas como:

  \begin{itemize}
    \item Desenvolvimento guiado por testes (TDD, do inglês \textit{Test-Driven Development})
    \item Desenvolvimento guiado por comportamento (BDD, do inglês \textit{Behaviour-Driven Development}
    \item Integração contínua
    \item Dublês de teste
  \end{itemize}

  Comparando as diferentes abordagens, possibilidades, pontos em aberto no emprego de cada técnica assim como observando a aplicabilidade de cada uma delas e a eficiência respectiva.

  A ideia inicial para desenvolver este trabalho foi causada pela vivencia do autor no estagio desenvolvido durante o período de estudante.....AQUI É COM VOCÊ HUGO..... Tivemos dificuldades na compreensão dos conceitos.... [HUGO: ainda não consegui pensar o que eu escreveria aqui.]

  [HUGO: A parte que explico de que se trata o kanban-roots seria eliminada, permaneceria aqui, ou entraria em um outro tópico?]

  Pretende-se como objetivo secundário deixar está monografia com aspecto de tutorial para iniciantes dos cursos de Ciência da Computação que se interessem por esta área de desenvolvimento, visto as dificuldades sofridas na busca de texto literário-conceitual e codificação prática destes conceitos juntamente. [HUGO: Isso não entraria na seção de justificativas e objetivos?]


6)

  DESENVOLVIMENTO/REFERENCIAL TEORICO

  Teste de Software é considerado por [Sommerville, 2007] uma técnica dinâmica de verificação e validação. Onde executamos uma implementação do software desenvolvido, apoiados em dados de teste, examinamos a saída e seu comportamento operacional para verificarmos se o seu desempenho está como solicitado/conforme necessário.

  Podemos inserir uma figura do processo de desenvolvimento iterativo, dado por Sommerville na figura 17.1 da pagina 261.

  [HUGO: em que seção essa parte entraria? Na fundamentação quando falamos de teste de software?]


7)

  Verificar qual a parte conceitual dos métodos ágeis (texto definido na página 12). [HUGO: não entendi]