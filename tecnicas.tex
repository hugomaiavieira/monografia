\chapter{Técnicas}

\section{Test-Driven Development (TDD)}
\label{sub:tdd}
``Apenas escrever código para corrigir um teste falhando". Segundo \citeonline{TestDrivenKoskela}, isto é \textit{Test-Driven Development} (Desenvolvimento guiado por testes) \cite{TDDbyExample} em apenas uma sentença.

O TDD é uma técnica onde o desenvolvimento do software é guiado por \textbf{testes automatizados}, que são escritos antes de qualquer linha de código relativo a funcionalidades. Primeiro escreve-se um teste, depois escreve-se o código para passar neste teste. Em seguida, o código é refatorado para encontrar um design melhor, contando sempre com os testes existentes para que não sejam introduzidas falhas em outras partes do sistema.

Esta abordagem encoraja bom design \cite{GrowingOOByTests}, produz código testável e mantém longe a sobre-engenharia por conta de falsas suposições, pois, nos testes, é especificado o que é desejado e escreve-se o código para fazer apenas aquilo que realmente é necessário. \cite{TestDrivenKoskela, TDDbyExample, EmpiricalTDD}

Mas TDD é uma técnica emergente? Não e sim.

O TDD vem sendo utilizado esporadicamente há anos, contudo, não existia um nome para identificar essa forma de desenvolver software. Além disso, em termos de adoção, TDD continua sendo novo \cite{TestDrivenKoskela, TDDbyExample, EmpiricalTDD}. Hoje, esta técnica tem um nome e começa a ganhar força, sendo utilizada em times de grandes empresas como Google, Yahoo, Microsoft e IBM \cite{EmpiricalTDD}.

\subsection{Ciclo TDD}
\label{ssub:ciclo_tdd}

Com base no trabalho de \citeonline{TDDbyExample}, o ciclo de desenvolvimento TDD é composto pelas seguintes etapas:

\begin{enumerate}
\item \textbf{Adicionar um teste}

Cada ciclo se inicia com a criação de um teste de unidade. Este teste inevitavelmente irá falhar, pois é escrito antes do código ser implementado de fato. Para escrever um teste, o desenvolvedor precisa entender claramente as especificações e requisitos da unidade. Isso faz com que o desenvolvedor tenha como foco os requisitos antes do código, o direcionando a escrever código apenas para o que é realmente necessário.

\item \textbf{Executar todos os testes e ver se algum falha}

Todos os testes devem ser executados e o novo teste deve falhar pela razão esperada: a funcionalidade não foi desenvolvida. Isto aumenta a confiança que se está testando a coisa certa.

\item \textbf{Escrever código}

O próximo passo é escrever código \textbf{somente para que o teste passe}. O código poderá não ser perfeito, pois posteriormente ele será melhorado. O importante é que o código faça o mínimo para passar no teste.

\item \textbf{Executar os testes e ter sucesso}

Ao Executar os testes e todos eles passarem, o código possuirá todos os requisitos testados e o programador pode ficar confiante para melhorá-lo.

\item \textbf{Refatorar}

Esta é uma etapa muito importante, onde o código escrito anteriormente é melhorado.

Segundo \citeonline{FowlerRefatoracao}, refatorar é reestruturar o software aplicando uma série de alterações em sua estrutura interna para torná-lo mais fácil de ser entendido e menos custoso de ser modificado, sem alterar seu comportamento observável.

Refatorar melhora o projeto do software, o torna mais fácil de entender e modificar, ajuda a encontrar falhas e ajuda o desenvolvedor a programar mais rapidamente.

Como na refatoração o comportamento do código não deve ser alterado, após refatorar e executar novamente os testes, todos eles devem passar.

\end{enumerate}

A figura \ref{img:ciclo-tdd} resume o ciclo TDD de forma bem clara.

\begin{figure}[h]
  \center
  \caption{O ciclo TDD}
  \includegraphics[scale=0.45]{images/ciclo-tdd}
  \label{img:ciclo-tdd}
\end{figure}

\subsection{TDD na prática}
\label{ssub:tdd_na_pratica}

Vejamos agora como isso se dá na prática através de um exemplo retirado do código do kanban-roots.

A seguinte funcionalidade precisa ser implementada:

\begin{quote}
\textit{Precisa-se saber todas as tarefas em uma determinada posição do kanban de um projeto.}
\end{quote}

A primeira coisa a ser feita é o teste para o caso mais simples: quando o projeto não tem tarefa alguma naquela determinada posição.

Desta forma, o teste para o caso mais simples pode ser como mostrado no código \ref{code:tdd_test1}.

\begin{mycode}{ruby}%
{Teste para o método Project\#tasks\_by\_position (versão 1)}{code:tdd_test1}
# test/unit/project_test.rb
class ProjectTest < ActiveSupport::TestCase
  def test_tasks_by_position
    project = Factory.create :project
    assert_equal(project.tasks_by_position(4), [])
  end
end
\end{mycode}

Ao rodar este teste, ele irá falhar, informando que o método \textit{tasks\_by\_position} sequer existe. Como esta era a falha esperada, escrevemos então o código mais simples para passar neste teste, apresentado no código \ref{code:tdd_code1}.

\begin{mycode}{ruby}%
{Código do método Project\#tasks\_by\_position (versão 1)}{code:tdd_code1}
# app/models/project.rb
def tasks_by_position position
  []
end
\end{mycode}

Os testes irão passar. Mas a funcionalidade ainda não está completa e, consequentemente, o teste também não. Como pode ser visto no código \ref{code:tdd_test2}, é adicionado ao teste uma verificação de que para a posição 1 devem existir duas tarefas e que estas tarefas devem ser exatamente as criadas anteriormente.

\begin{mycode}{ruby}%
{Teste do método Project\#tasks\_by\_position (versão 2)}{code:tdd_test2}
# test/unit/project_test.rb
class ProjectTest < ActiveSupport::TestCase
  def test_tasks_by_position
    project = Factory.create :project
    tasks = [Factory.create(:task, :project => project, :position => 1),
             Factory.create(:task, :project => project, :position => 1)]

    assert_equal(project.tasks_by_position(4), [])

    assert_equal(project.tasks_by_position(1).count, 2)
    tasks.each { |task| assert(project.tasks_by_position(1).include?(task)) }
  end
end
\end{mycode}

Como o teste foi alterado, este é executado e irá falhar, informando que na \hyperref[code:tdd_test1]{linha 10} eram esperadas duas tarefas, mas foram obtidas zero. O código então deve ser modificado para passar no novo teste e é apresentado no código \ref{code:tdd_code2}.

\begin{mycode}{ruby}%
{Código do método Project\#tasks\_by\_position (versão 2)}{code:tdd_code2}
# app/models/project.rb
def tasks_by_position position
  task_list = []
  tasks.each do |task|
    task_list << task if task.position.to_s == position.to_s
  end
  task_list
end
\end{mycode}

Desta vez, após a modificação no código e a execução dos testes, todos os testes os irão passar. Contudo, a cobertura de testes para este método ainda está fraca, o que pode ser resolvido com a adição de mais algumas tarefas em posições diferentes, como mostrado no código \ref{code:tdd_test3}.

\begin{mycode}{ruby}%
{Teste do método Project\#tasks\_by\_position (versão 3)}{code:tdd_test3}
# test/unit/project_test.rb
class ProjectTest < ActiveSupport::TestCase
  def test_tasks_by_position
    project = Factory.create :project
    tasks = [Factory.create(:task, :project => project, :position => 1),
             Factory.create(:task, :project => project, :position => 1),
             Factory.create(:task, :project => project, :position => 2),
             Factory.create(:task, :project => project, :position => 3),
             Factory.create(:task, :project => project, :position => 3)]

    assert_equal(project.tasks_by_position(4), [])

    assert_equal(project.tasks_by_position(1).count, 2)
    tasks[0..1].each { |task| assert(project.tasks_by_position(1).include?(task)) }

    assert_equal(project.tasks_by_position(2), [tasks[2]])

    assert_equal(project.tasks_by_position("3").count, 2)
    tasks[3..4].each { |task| assert(project.tasks_by_position("3").include?(task)) }
  end
end
\end{mycode}

Executando os testes novamente, todos eles passam. Isto indica que já é hora de ir para o item 5 do ciclo TDD e refatorar. Ao comparar o Código \ref{code:tdd_code2} com o Código \ref{code:tdd_code3}, pode-se perceber como o Código \ref{code:tdd_code3} está mais simples e claro e que ao rodar os testes, estes passam mostrando que o comportamento do método não mudou.

\begin{mycode}{ruby}%
{Código do método Project\#tasks\_by\_position (versão 3)}{code:tdd_code3}
# app/models/project.rb
def tasks_by_position position
  tasks.select { |item| item.position.to_s == position.to_s }
end
\end{mycode}

\citeonline{TDDAntiPatterns} descreve vários anti-padrões do TDD e sua leitura é fortemente recomendada.

\subsection{Design emergente} % (fold)
\label{sub:design_emergente}

No contexto em que TDD costuma ser aplicado - em processos ágeis, com um ciclo de vida iterativo-incremental - não é realizado um amplo \textit{design} prévio. Sendo assim, o \textit{design} evolui incrementalmente. Sabendo disso, uma característica extremamente importante do TDD é a facilitação ao \textit{design} emergente, que muitos consideram ser sua a principal característica.

\textbf{TODO: dar um exemplo do kanban e PROCURAR REFERÊNCIAS SOBRE ISSO}

% subsection design_emergente (end)


\section{Behaviour-Driven Development (BDD)}
\label{sec:bdd}

Criado em 2006 \cite{IntroducingBDD}, \textit{Behaviour-Driven Development} (Desenvolvimento guiado por comportamento) é uma técnica de desenvolvimento de software cuja amplitude se estende às atividades de design, documentação, validação e verificação, tratando-as de modo unificado \cite{BDDRodrigo}.

\subsection{A herança de TDD}
\label{sub:a_heranca_de_tdd}

O BDD é uma evolução do TDD. A grande diferença entre os dois é que TDD não abrange a validação do software, ou seja, se o software atende os requisitos. Isso muda tudo, pois em BDD, a ênfase está no comportamento, e não na estrutura. Ao invés de focar em classes e métodos ao escrever os testes, o foco é no comportamento que gera valor para o sistema. Além disso, o pensamento não é voltado à verificação, mas sim no comportamento, em validar que o software faz sem deixar também de verificar se está funcionando como deveria.

Em BDD não são escritos testes, mas sim especificações, chamadas de \textit{specs} (de \textit{specifications}). Pode-se ter a clara diferença comparando o Código \ref{code:tdd_test3} com o Código \ref{code:bdd_spec}, onde palavras chave como \textit{class}, \textit{test} e \textit{assert} dão lugar a \textit{describe}, \textit{it} e \textit{should}, e fica nítido o foco no comportamento, inclusive nas asserções.

\begin{mycode}{rspec}%
{Spec}{code:bdd_spec}
# spec/models/project.rb
describe Project do
  it "returns all related tasks matching a given position" do
    project = Factory.create :project
    tasks = [Factory.create(:task, :project => project, :position => 1),
             Factory.create(:task, :project => project, :position => 1),
             Factory.create(:task, :project => project, :position => 2),
             Factory.create(:task, :project => project, :position => 3),
             Factory.create(:task, :project => project, :position => 3)]

    project.tasks_by_position(4).should be_empty

    project.should have(2).tasks_by_position(1)
    project.tasks_by_position(1).should include(*tasks[0..1])

    project.tasks_by_position(2).should == [tasks[2]]

    project.should have(2).tasks_by_position(3)
    project.tasks_by_position(3).should include(*tasks[3..4])
  end
end
\end{mycode}

Em TDD, o nome dos testes é baseado na estrutura do código, e isso pode ter um efeito colateral: criar duplicações. Ao escrever testes como o do Código \ref{code:tdd_test3} \textit{test\_tesks\_by\_position} e o método \textit{tasks\_by\_position} é renomeado, o nome do teste também terá que ser renomeado, o que provavelmente é esquecido. Isso torna os testes confusos e desinformativos \cite{ContinuousTesting}. Como por definição, refatorar é melhorar a estrutura e o design do código sem alterar seu comportamento, nomeando os testes com base no comportamento desejado ao invés da estrutura do código, não apenas torna os testes mais informativos como também torna a refatoração menos custosa \cite{ContinuousTesting}.

% subsection a_heranca_de_tdd (end)

\subsection{O ciclo BDD}
\label{sub:o_ciclo_bdd}

O Ciclo BDD, apresentado na Figura \ref{img:ciclo-bdd}, tem dois níveis: unidade e aceitação. O nível aceitação é o nível mais alto, onde são escritos os testes de aceitação. Já o nível unidade é o mesmíssimo ciclo TDD visto anteriormente.

\begin{enumerate}
\item \textbf{Adicionar um teste de aceitação}

Cada ciclo se inicia com a criação de um teste de aceitação, um cenário, que descreve um determinado comportamento de uma funcionalidade do sistema.

\item \textbf{Executar todos os testes e ver se algum falha}

Todos os testes devem ser executados e o novo teste deve falhar pela razão esperada: a funcionalidade não foi desenvolvida. Isto aumenta a confiança que se está testando a coisa certa.

\item \textbf{Descer de nível}

Neste momento, deve-se descer de nível, saindo do nível de aceitação e indo para  nível de unidade.

\item \textbf{Entra no ciclo TDD}

No nível de unidade, entra-se o ciclo TDD até que todos os testes de unidade estejam passando.

\item \textbf{Retornar para o nível de aceitação}

Com todos os testes de unidade passando, retorna-se para o nível de aceitação e faz-se todos os testes passarem.

\item \textbf{Refatorar}

Refatora-se o código.

\end{enumerate}

\begin{figure}[h]
  \center
  \caption{O ciclo BDD}
  \includegraphics[scale=0.45]{images/ciclo-bdd}
  \label{img:ciclo-bdd}
\end{figure}

% subsection o_ciclo_bdd (end)

\subsection{Modelos de escrita dos testes de aceitação}
\label{sub:modelos_de_escrita_dos_testes_de_aceitacao}

Existem duas maneiras de se escrever os testes de aceitação: em texto plano e em código puro. Para exemplificar cada um das abordagens, será especificada a seguinte funcionalidade:

\begin{quote}
\textit{Os comentários mostrados na página de tarefas devem ser renderizados utilizando a linguagem de marcação Markdown.}
\end{quote}

\subsubsection{Escrita em texto plano} % (fold)
\label{subsub:escrita_em_texto_plano}

Uma das maneiras de escrever testes de aceitação é como texto plano, ou seja, inglês, português e etc. Nas ferramentas que implementam esta abordagem, os testes são escritos baseados em \textit{steps} (passos), onde cada step é mapeado para um código real.

No código \ref{code:bdd_cucumber_spec}, é utilizado o \textit{cucumber} para fazer a especificação. Como pode-se ver, o teste é escrito em linguagem que qualquer pessoa, mesmo sem nenhum conhecimento em programação, pode ler e validar.

\begin{mycode}{cucumber}%
{Especificação em texto plano}{code:bdd_cucumber_spec}
# features/comments.feature
Feature: Render comments with Markdown syntax
  As a user
  I want to use Markdown in my comments
  In order to make my comments more expressives

  Scenario: on tasks page
    Given I am a contributor of "sgtran" project
    And I am authenticated
    And I have a task of "sgtran" project
    When I am on the task page
    And I fill in "comment_content" with "# Some content [link](http://exemplo.com)"
    And I press "Comment"
    Then I should see "Some content" in a "h1" tag
    And I should see "link" in an "a" tag
\end{mycode}

Para cada \textit{step} existe uma implementação, ou seja, sua definição, chamada de \textit{step definition}. Pode-se pensar nos \textit{steps} como chamadas à métodos, e assim, os \textit{step definitions} serão as definição desse métodos.Os \textit{step definitions} para os \textit{steps} do código \ref{code:bdd_cucumber_spec} são apresentado no código \ref{code:step_definition}.

Como durante o desenvolvimento são escritos muitos \textit{step definitions}, com o intuito de organizá-los, eles são separados por contexto em arquivos diferentes.

\begin{mycode}{ruby}%
{Step definitions}{code:step_definition}
# features/step_definitions/contributor_steps.rb
Given /^I am a contributor of "([^"]*)" project$/ do |name|
  @project = Factory.create :project, :name => name
  @contributor = Factory.create :contributor, :contributions => [@project]
  @project.update_attribute(:contributors, [@contributor])
end

# features/step_definitions/contributor_steps.rb
Given /^I am(?:| an) authenticated(?:| contributor)$/ do
  @contributor ||= Factory.create :contributor

  Given %{I am on the sign in page}
  And %{I fill in "contributor_login" with "#{@contributor.email}"}
  And %{I fill in "contributor_password" with "#{@contributor.password}"}
  And %{I press "Sign in"}
end

# features/step_definitions/task_steps.rb
Given /^I have a task of "([^"]*)" project$/ do |project_name|
  @project = Factory.create :project, :name => 'project_name'
  @task = Factory.create :task, :project => @project, :author => @contributor
end

# features/step_definitions/web_steps.rb
Given /^(?:|I )am on (.+)$/ do |page_name|
  visit path_to(page_name)
end

# features/step_definitions/web_steps.rb
When /^(?:|I )fill in "([^"]*)" with "([^"]*)"$/ do |field, value|
  fill_in(field, :with => value)
end

# features/step_definitions/web_steps.rb
When /^(?:|I )press "([^"]*)"$/ do |button|
  click_button(button)
end

# features/step_definitions/my_web_steps.rb
When /^I should see "([^"]*)" in a(?:|n) "([^"]*)" tag$/ do |text, tag|
  if page.respond_to? :should
    page.should have_xpath("//#{tag}", :text => text)
  else
    assert page.has_xpath?("//#{tag}", :text => text)
  end
end
\end{mycode}

% subsubsection escrita_em_texto_plano (end)


\subsubsection{Escrita em código puro} % (fold)
\label{subsub:Escrita em codigo puro}

A outra maneira de escrever os testes de aceitação é com código puro. Nesta forma, todo o código dos testes está concentrado em apenas um lugar

No código \ref{code:bdd_spec1} pode-se ver a especificação em código puro para a funcionalidade.

\begin{mycode}{rspec}%
{Especificação}{code:bdd_spec1}
# spec/acceptance/comments_spec.rb
feature "Render comments with Markdown syntax" do
  background do
    @owner = Factory.create :contributor
    @project = Factory.create :project
    @task = Factory.create :task, :project => @project, :author => @owner
    login(@owner.email, @owner.password)
  end

  scenario "on tasks page" do
    visit project_task_path(@project, @task)
    fill_in "comment_content", :with => "# Some content [link](http://exemplo.com)"
    click_button "Comment"
    page.should have_xpath("//h1", :text => "Some content")
    page.should have_xpath("//a", :text => "link", :href => "http://exemplo.com")
  end
end
\end{mycode}
% subsubsection Escrita em codigo puro (end)

% subsection modelos_de_escrita_dos_testes_de_aceitacao (end)

\subsection{Questões em aberto} % (fold)
\label{sub:questoes_es_em_aberto}

BDD vem se tornando um consenso em automação de testes. Contudo, existe uma divergência grande sobre a forma de escrita dos testes de aceitação. Muitos defendem que os testes devem ser escritos em texto plano, por conta do teste teste ser escrito em linguagem que qualquer pessoa, mesmo sem nenhum conhecimento em programação, pode ler, validar e inclusive escrever. Existem relatos de que na Globo.com esta abordagem é utilizada \textbf{(TODO: procurar referências e outros casos)} e os clientes escrevem os testes de aceitação.

No entanto, muitos veem essa abordagem como utópica \textbf{(TODO: procurar referências)}, e que nenhum cliente vai ler, muito menos escrever testes.

Outra critica aos testes em texto plano é que enquanto nos testes em código puro todo o código do teste está em concentrado em apenas um lugar, nos testes em texto plano, o código do testes está espalhado em diversos arquivos e métodos, devido a camada adicional dos \textit{steps}, tonando a \textit{suite} de testes mais complexa de manter e estender \textbf{(TODO: procurar referências)}.

Além disso, outro questionamento é sobre a velocidade para a execução dos testes. Como o \textit{Cucumber} utiliza a \textit{Gherkin}\footnote{\url{http://github.com/cucumber/cucumber/wiki/Gherkin}}, que é uma DSL externa\footnote{\textit{Domain specific language} (linguagem específica de domínio) é uma linguagem de programação de expressividade limitada, focada num domínio particular. Uma DSL externa tem sua própria sintaxe e necessita de um \textit{parser} para processá-la \cite{DSLFowler}.}, o arquivo de \textit{features} é parseado para que os \textit{steps} sejam executados, introduzindo um tempo extra na execução dos testes.

O exemplo apresentado nos itens anteriores foram executados separadamente e seus tempos de execução medidos, sendo os resultados apresentados na tabela \ref{table:tempo_de_execucao}. Analisando estes resultados pode-se concluir que os testes em código puro executam, em média, 60,56\% mais rápido que os em texto plano.

\begin{table}[ht]
\caption{Velocidade de execução dos testes de acordo com o método de escrita}
\label{table:tempo_de_execucao}
\centering
\begin{tabular}{p{4.5cm} p{6.5cm}}
\toprule
\textbf{Método de escrita} & \textbf{Tempo médio de execução} \\
\midrule[1pt]
texto plano & 0m0.284s \\ \midrule
código puro & 0m0.172s \\
\bottomrule
\end{tabular}
\end{table}

Na opinião do autor, os testes escritos em texto plano são muito bonitos, porém, pouco produtivos. TODO: fazer uma conclusão descente sobre esse tópico.

% subsection questoes_es_em_aberto (end)

% section bdd (end)

\section{Integração contínua}

Em uma equipe com vários desenvolvedores, todos trabalhando na elaboração de um mesmo sistema, existe o problema de unificar as diversas alterações feitas na base de código, assegurando que a base continua consistente \cite{ImproveitCI}. Para resolver esse problema, entra em cena a Integração Contínua (IC), que além disso, tem como ponto chave dar um feedback rápido quando a base não está consistente.

\cite{FowlerCI} definiu a IC da seguinte maneira:

\begin{citacao}
Integração Contínua é uma prática de desenvolvimento de software onde os membros de um time integram seu trabalho frequentemente, geralmente cada pessoa integra ao menos uma vez ao dia - podendo haver múltiplas integrações por dia. Cada integração é verificada por um \textit{build} automatizado (incluindo testes) para detectar erros de integração o mais rápido possível. Muitos times acham que essa abordagem leva a uma significante redução nos problemas de integração e permite que um time desenvolva software coeso mais rapidamente.
\end{citacao}

A IC é um dos pilares da agilidade, pois garante que todo o sistema funcione de forma coesa a cada \textit{build}, mesmo que sua equipe seja grande e diversas partes do código estejam sendo alteradas ao mesmo tempo \cite{CaelumCI}.

Existe um debate sobre a periodicidade da integração, que tem relação direta com o tempo de execução da \textit{build}. Para assegurar o rápido feedback, esse tempo de execução deve ser o menor possível, tentando manter sempre menor do que dez minutos \cite{FowlerCI}.

\section{Teste contínuo}
\label{sub:teste_continuo}

Falar aqui de continuous testing.

\section{Dublês de Teste}

Em algumas ocasiões é difícil testar alguns componentes porque eles dependem de outros componentes que não podem ser utilizados em ambiente de teste. Estas situações podem acontecer por esses componentes não estarem disponíveis, por eles não retornarem os resultados necessários ou porque executá-los iria trazer efeitos colaterais indesejados. Em outros casos, a estratégia de testes utilizada requer que se tenha mais controle do comportamento interno do componente.

Quando se escreve um teste onde não se pode/escolhe usar componentes reais, pode-se substitui-los pelos Dublês de Teste que oferecem uma maneira de isolar as dependências ao criar os testes, permitindo a utilização de componentes falsos para cumprir os papéis de componentes reais. Com isso, elimina-se complexidade do código dos testes, pois o código de implementação dos objetos é mantido pequeno e com baixo acoplamento.

Os Dublês de Teste não precisam se comportar exatamente como o componente real, devendo apenas prover a mesma API que o componente real.

\citeonline{XUnit} define cinco categorias de Dublês de Teste:

\begin{itemize}
\item
Objetos \textbf{\textit{Dummy}} geralmente são utilizados apenas para preencher uma lista de parâmetros e nunca são realmente usados.

\item
Objetos \textbf{\textit{Fake}} são utilizados para substituir funcionalidades reais de um componente por razões diferentes de verificações indiretas de entradas e saídas do componente a ser   testado.

\item
Os \textbf{\textit{Stubs}} provêm respostas prontas para chamadas feitas durante os testes, geralmente não respondendo a qualquer   chamada diferente
das pré-definidas.

\item
Os \textbf{\textit{Spies}} são \textit{Stubs} que também tem gravam algumas informações baseadas em como eles são chamados. Um exemplo   pode ser um serviço de email que grava quantas mensagens foram   enviadas.

\item
\textbf{\textit{Mocks}} são objetos pré-programados para receber determinado conjunto de chamadas, podendo lançar uma exceção se tais chamadas não forem feitas a ele, ou se receber outra chamada diferente das pré-programadas.

referência: the art of agile development, página 298.
\end{itemize}

De modo geral, a utilização de Dublês de Teste é extremamente benéfica para o projeto. Contudo, existem controvérsias, principalmente em relação a utilização do \textit{Mock} \cite{MocksArentStubs}.
